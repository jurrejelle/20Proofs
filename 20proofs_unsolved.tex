\documentclass{article}
\usepackage[utf8]{inputenc}
\usepackage{amssymb}
\title{20 proofs in 1 sitting}
\author{J.J.K Groenendijk}
\date{26/10/2020}
\begin{document}

\section{Proof 1}
for $n \in N: 1+2+4+...+2^n = 2^{(n+1)}-1$\\


\section{Proof 2}
For all positive integers $x, y:$  $ (x\mid y) \rightarrow (x \mid y^x)$.\\

\section{Proof 3}


Assuming you have two positive numbers a, b and the equation $a^2 + 2a = 4b^2 + 4b$ holds, prove that $a = 2b$.\\



\section{Proof 4}
For integers a,b, there are no integers s.t. $a^2-9b=3$\\

\section{Proof 5}
If $x^2-8x+11$ is even then x is odd\\

\section{Proof 6}

For any $n \in N \land n \geq 1$, $1+3+5+7+...+2n-1 = n^2$\\


\section{Proof 7}
Proof that $A \cup (A \cap B)$ = A for any sets A,B\\


\section{Proof 8}
Say $A [] B = \{x \mid x \in A \oplus x \in B\}$, show that $A [] B = (A - B) \cup (B - A)$\\

\section{Proof 9}
$$\forall n\geq 1, n \in N : \sum^{n}_{i=1}i(i+1) = \frac{n(n+1)(n+2)}{3}$$\\

\section{Proof 10}
We recursively define S. Say $(0,0) \in S$. \\Recursion: $(m,n) \in S, \rightarrow (m+2,n+3) \in S$. \\There are no other elements in S. \\Prove that $ (m,n) \in S \rightarrow 5 \mid (m+n)$\\

 
\section{Proof 11}
For any set A,B,C: prove that $(A \cup B) - C = (A - C) \cup (B - C)$\\



\section{Proof 12}
For integers x,y,z,n prove that  $x^n+y^n=z^n$ has no solutions for n greater than 2\\


\section{Proof 13}
Define set S like this: $1 \in S,\\ x \in S \rightarrow x+3 \in S, \\2\mid x \in S \rightarrow \frac{x}{2} \in S$\\ And nothing else.\\
Proof $k \in S \rightarrow 3 \nmid k$ \\




\section{Proof 14}
 $a \in S, \\x \in S \rightarrow xb \in S, \\x \in S \rightarrow abxa \in S, \\x,y \in S \rightarrow xya \in S$\\
There are no other elements in S.\\
Prove that the number of a is odd in every string\\

\section{Proof 15}
Let A be the set of all bitstrings of form $0^n1^n$ with $n \in N$, \\
Let B be the recursively defined set $01 \in B, \\w \in B \rightarrow 0w1 \in B$\\ There are no other elements in B.\\ prove A=B\\


\section{Proof 16}
for all $n \in N, 3 \mid 5^{2n}-1$\\


\section{Proof 17}
$A \cap B \subseteq A \subseteq A \cup B$ for any sets A,B\\


\section{Proof 18}
for all natural numbers $n\geq1$ $$(\sum^{n}_{i=1}i)^2 = \sum^{n}_{i=1}i^3$$ (HH 4.2b)\\

\section{Proof 19}
x is odd iff $\mid x\mid$ is odd for any integer x.\\

\section{Proof 20}
suppose x and y have opposite parity, xy should be even, with $x,y \in Z$\\


 
\end {document}
